\documentclass[a5paper,
              12pt,
              pagesize,
              typearea,
              titlepage,
              DIV=10,
              twoside
              ]{scrbook}


\usepackage[utf8]{inputenc}
\usepackage[T1]{fontenc}
\usepackage[ngerman]{babel}
              

%Zusätzliche Beschnittzugabe zum DinA6 Format dazu
\usepackage[cam, noinfo, width=15.4truecm,height=21.6truecm, center]{crop}

%Überschriftenabstände
\usepackage{titlesec}

%Kopf- und Fußzeilen
\usepackage{fancyhdr}

%Schrift
\usepackage[sfdefault]{AlegreyaSans}

%Gitarrenakkorde
\usepackage{guitar}

%Index
\usepackage{makeidx}

%Strophennummerierung
\usepackage{enumitem}

%PDF-Dateien einbinden 
\usepackage{pdfpages}

%Musiknoten schreiben
\usepackage[generate,ps2eps]{abc}

%Layoutgeraffel
%may the latex masters be gentle and hope that they won't take this clusterf**k personal - but it works
%setting up the layout spacing 
\areaset[10mm]{130 mm}{187 mm}
\topmargin=-21.4mm 
\headheight=5.5mm
\headsep=2.5mm
\footskip=6mm
\setlength{\parindent}{0em}
\setlist[enumerate]{topsep=0pt, itemsep=5pt, partopsep=-10pt, parsep=5pt, wide, labelwidth=!, labelindent=0pt, label=\bfseries{\arabic*.}}
\setcounter{secnumdepth}{0}

\setlength\parskip{1\baselineskip plus 1\baselineskip minus 0.15\baselineskip}

\titleformat{\chapter}[display]  
{\normalfont\huge\bfseries}{\chaptertitlename\ \thechapter}{15pt}{\Huge}   

\titleformat{\section}[display]
{\normalfont\normalfont\bfseries}{\chaptertitlename\ \thechapter}{15pt}{\Large}

\titlespacing*{\chapter}{0pt}{-15mm}{5mm}
\titlespacing*{\section}{0mm}{0mm}{-5mm}
\fancypagestyle{plain}{\fancyhf{}%
  \renewcommand{\headrulewidth}{0pt}
}
\renewcommand{\footrulewidth}{0.4pt}% default is 0pt
\renewcommand{\headrulewidth}{0.4pt}% default is 0pt

%Befehlesetup
\newcommand{\whzl}{\raisebox{-1mm}{\hspace{0.3mm}\rule{0.3mm}{4mm}\hspace{0.3mm}\rule{0.1mm}{4mm}\hspace{0.2mm}}:}
\newcommand{\whzr}{:\raisebox{-1mm}{\hspace{0.2mm}\rule{0.1mm}{4mm}\hspace{0.3mm}\rule{0.3mm}{4mm}}}
\newcommand{\liedweiter}{ \AddToShipoutPictureFG*{%
                            \AtTextLowerLeft{%
                              \makebox[\textwidth][r]{
                                  \includegraphics[scale=0.025]{icons/wegzeichen_naechste.pdf}
                              }%
                            }%
                          }
                          \newpage 
}
\newcommand{\LiedSetup}{ 
                        \noindent
                        \section{\Titel}
                        \fancyhead[RE,LO]{\nouppercase{\Titel}}
                        \fancyhead[RO,LE]{\small{\nouppercase{\Interpret}}} 
                        \fancyfoot[LE,RO]{\small{\nouppercase{\Referenz}}} 
                        \fancyfoot[LO,RE]{\thepage}
}
\newcommand{\GeistlichSetup}{
                        \noindent
                        \section{\Titel}
                        \fancyhead[RE,LO]{\nouppercase{\Titel}}
                        \fancyhead[RO,LE]{\small{\nouppercase{\Interpret}}} 
                        \fancyfoot[LE,RO]{\small{\nouppercase{\Referenz}}} 
                        \fancyfoot[LO,RE]{\thepage}
                        \addcontentsline{lof}{section}{\Titel}
}                            

\makeindex

\begin{document}
  \pagestyle{empty}
  

    %Cover 
    \includepdf[scale=2.1]{bilder/klappentext_modern.pdf}
    %leerer Klappentext
    \cleardoublepage
    
    %Hier ein Vorwort, welches nicht im Inhaltsverzeichnis auftaucht
    \chapter*{Vorwort}
      Hi,

      ich bin das Vorwort von dem Liederbuchtemplate das du gerade geöffnet hast.
      Hier könntest du dich allen zukünftigen Lesern deines Liederbuches vorstellen und mitteilen was für großartige Abende sie damit verbringen können.
      
      Gezeichnet,

      dein Liederbuchtemplate

      \newpage

    %Das Inhaltsverzeichnis
    \tableofcontents
    \newpage
 

    %Seitenrahmen einrichtung
    \pagenumbering{arabic}
    \fancyhf{}
    \pagestyle{fancy}
    
    % Dieses Template enhält nur zwei Lieder. Wegen Liedrechten und solchen Dingen
    \def\Titel{Die Gedanken sind frei}
\def\Interpret{Deutsches Volkslied}
\def\Referenz{Möglicher Querverweis auf ein gebräuchliches Liederbuch deiner Wahl}

% hier entweder \LiedSetup() oder \GeistlichSetup{} wenn es im Kuratenverzeichnis auftauchen soll
\LiedSetup{}

\begin{guitarMagic}
	\begin{enumerate}

        \item Die Ge[A]danken sind frei, wer [E7]kann sie er[A]raten? \\
        Sie fliehen vorbei wie [E7]nächtliche [A]Schatten.\\
        Kein [E]Mensch kann sie [A]wissen, kein [E]Jäger er[A]schießen
        mit [D]Pulver und [A]Blei.\\
        Die Ge[E]danken sind [A]frei!

        \item Ich [A]denke, was ich will und [E]was mich be[A]glücket,\\
        doch alles in der Still’
        und [E]wie es sich [A]schicket.\\
        Mein [E]Wunsch und Be[A]gehren
        kann [E]niemand ver[A]wehren,
        es [D]bleibet da[A]bei:\\
        Die Ge[E]danken sind [A]frei!

        \item Und [A]sperrt man mich ein
        im [E7]finsteren [A]Kerker,\\
        das alles sind rein
        ver[E7]gebliche [A]Werke,\\
        denn [E]meine Ge[A]danken
        zer[E]reißen die [A]Schranken
        und [D]Mauern ent[A]zwei.\\
        Die Ge[E]danken sind [A]frei!

        \item Drum [A]will ich auf immer
        den [E7]Sorgen ent[A]sagen\\
        und will mich auch nimmer
        mit [E7]Grillen mehr [A]plagen.\\
        Man [E]kann ja im [A]Herzen
        stets [E]lachen und [A]scherzen
        und [D]denken da[A]bei:\\
        Die Ge[E]danken sind [A]frei!

        \item Ich [A]liebe den Wein,
        mein [E7]Mädchen vor [A]allen,\\
        sie tut mir allein
        am [E7]besten ge[A]fallen.
        
        % So wird ein Seitenumbruch mit dem "auf der nächsten Seite gehts weiter" eingefügt
        \liedweiter{}

        Ich [E]bin nicht al[A]leine
        bei [E]meinem Glas [A]Weine,
        mein [D]Mädchen da[A]bei.\\
        Die Ge[E]danken sind [A]frei!

    \end{enumerate}
\end{guitarMagic}

% Bilder können nicht in der guitarmagic Umgebung eingefügt werden. Das muss davor oder danach geschehen
\begin{picture}(0,0)
    \put(-5,-150){\hbox{\includegraphics[scale=0.25, angle=0]{bilder/Flaschenpost.png}}}
\end{picture}

 % Beispiel Gitarren Akkorden
    \def\Titel{Die Internationale}
\def\Interpret{Melodie: Pierre De Geyter (1888)  Text: Emil Luckhardt (1919)}
\def\Referenz{Möglicher Querverweis auf ein gebräuchliches Liederbuch deiner Wahl}

% hier entweder \LiedSetup() oder \GeistlichSetup{} wenn es im Kuratenverzeichnis auftauchen soll
\LiedSetup{}

\begin{abc}[name=DieInternationale]

X: 1
M: C % Takt
K: G % Tonart
"G"D2 | G3 F A G D B, | "C"E4 C2 E2 | "Am"A3 G "D7"F E D C | "G"B,4 z2 D |   % Noten
w: 1.~~Wacht auf, Ver-damm-te die- ser Er- de,                % Text
+: die stets man noch zum Hun- gern zwingt! Das
"G"G3 F A G D B, | "C"E4 C2 "Am"a G | "D"F2 A2 "D7"c2 F2 | "G"G4 z2 B A |
w: Recht wie Glut im Kra- ter- her- de
+: nun mit Macht zum Durch- bruch dringt. Rei- nen
"D"F3 F "A7"(EF) G E | "D"F4 D2 ^C D | "A7"E3 E A3 G | "D"F4 z2 A2 |
w: Tisch macht mit* dem Be- drän- ger!
+: Heer der Skla- ven, wa- che auf! Ein
    "D7"A3 G D D ^C D | "G"B4 "(Em)"G E F G | "D"F2 A2 "A7"G2 E2 | "D"F4 z2 "D7"B>A |
w: Nichts zu sein, tragt es nicht län- ger,
+: al- les zu wer- den, strömt zu- hauf! Völ- ker,
    "G"G4 D3 B, | "C"E4 C2 "(Am)"A>G | "D"F4 E3 D | "G"D4 z2 D | B3 B "D"A2 D2 |
w: hört die Sig- na- le! Auf zum letz- ten Ge- fecht!
+: Die In- ter- na- tio-
    "G"G4 "D"F3 F | "A"E3 ^D E2 "A7"A2 | "D"A4 z2 B>A | "G"G4 D3 B, | "C"E4 C2 "(Am)"A>G |
w: na- le er- kämpft das Men- schen- recht. Völ- ker, hört die Sig- na- le! Auf zum
"D"F4 E3 D | "G"B4 z2 B | d3 d c2 B2 | "D"(A2B2) c2 z c | "G"B3 G "D"A3 F | "G"G4 z2 |]
w: letz- ten Ge- fecht!
+: Die In- ter- na- tio- na-- le
+: er- kämpft das Men- schen- recht.

\end{abc}

\begin{guitarMagic}
    \begin{enumerate}
        \setcounter{enumi}{1} % Starte bei 2 da die erste Strophe bei den Noten steht.
        \liedweiter

        \item Es [G]rettet uns kein höh’res W[C]es[(AM)|]{en,}
            kein G[D]ott, kein K[D7]aiser noch Tri[G]bun \\
            Uns [G]aus dem Elend zu erl[C]{\"o}s[(Am)]en 
            können wi[D]ir nur s[D7]elber t[G]un! \\
            Leeres W[D]ort: des [A7]Armen R[D]echte, 
            Leeres W[D]ort: des [A7]Reichen Pfl[D]icht! \\
            Unm[D7]{\"u}ndig nennt man uns und Kn[G]echte, 
            duldet die Schm[D]ach nun l[A7]{\"a}nger n[D]icht! [D7]

        \textbf{Refrain:}
            \leftrepeat Völker, h[G]{\"o}rt die Sig[C]na[(Am)|]{le!}
            Auf zum l[D]etzten Gef[G]echt! \\
            Die Intern[D]ation[G]al[D]e
            erk[G]{\"a}mpft das M[D]enschenr[G]echt. \rightrepeat


        \item In St[G]adt und Land, ihr Arbeitsl[C]eu[(Am)|]{te,}
            w[D]ir sind die st[D7]{\"a}rkste der Part[G]ei’n \\
            Die M[G]{\"u}{\ss}iggänger schiebt beis[C]ei[(Am)|]{te!}
            Diese W[D]elt muss [D]unser s[G]ein; \\
            Unser Bl[D]ut s[A7]ei nicht mehr der R[D]aben, 
            Nicht der m[A]{\"a}cht’gen G[A7]eier Fr[D]aß! \\
            Erst w[D7]enn wir sie vertrieben h[G]aben 
            dann scheint die S[D]onn’ ohn’ [A7]Unterl[D]ass! [D7]

        \textbf{Refrain:}
            \leftrepeat Völker, h[G]{\"o}rt die Sig[C]na[(Am)|]{le!}
            Auf zum l[D]etzten Gef[G]echt! \\
            Die Intern[D]ation[G]al[D]e
            erk[G]{\"a}mpft das M[D]enschenr[G]echt. \rightrepeat


    \end{enumerate}
\end{guitarMagic}

\notebox{Die Internationale ist ein sozialistisches Arbeiterlied. Ursprünglich wurde das Lied auf französisch geschrieben und später ins Deutsche übersetzt.}

 % Beispiel mit Noten
    

    \pagestyle{empty}

    %Das Kuratenverzeichnis fasst alle entsprechend markierten Lieder in einem weiteren Inhaltsverzeichnis zusammen
    \renewcommand\listfigurename{Kuratenverzeichnis}
    \addcontentsline{toc}{chapter}{\listfigurename{}}
    \listoffigures
    \printindex


    %Leerer hinterer Klappentext
    \cleardoublepage
    %Letzte Seite
    \includepdf[scale=2.1]{bilder/klappentext_modern.pdf}
    
\end{document}

