\def\Titel{Die Internationale}
\def\Interpret{Melodie: Pierre De Geyter (1888)  Text: Emil Luckhardt (1919)}
\def\Referenz{Möglicher Querverweis auf ein gebräuchliches Liederbuch deiner Wahl}

% hier entweder \LiedSetup() oder \GeistlichSetup{} wenn es im Kuratenverzeichnis auftauchen soll
\LiedSetup{}

\begin{abc}[name=DieInternationale]

X: 1
M: C % Takt
K: G % Tonart
"G"D2 | G3 F A G D B, | "C"E4 C2 E2 | "Am"A3 G "D7"F E D C | "G"B,4 z2 D |   % Noten
w: 1.~~Wacht auf, Ver-damm-te die- ser Er- de,                % Text
+: die stets man noch zum Hun- gern zwingt! Das
"G"G3 F A G D B, | "C"E4 C2 "Am"a G | "D"F2 A2 "D7"c2 F2 | "G"G4 z2 B A |
w: Recht wie Glut im Kra- ter- her- de
+: nun mit Macht zum Durch- bruch dringt. Rei- nen
"D"F3 F "A7"(EF) G E | "D"F4 D2 ^C D | "A7"E3 E A3 G | "D"F4 z2 A2 |
w: Tisch macht mit* dem Be- drän- ger!
+: Heer der Skla- ven, wa- che auf! Ein
    "D7"A3 G D D ^C D | "G"B4 "(Em)"G E F G | "D"F2 A2 "A7"G2 E2 | "D"F4 z2 "D7"B>A |
w: Nichts zu sein, tragt es nicht län- ger,
+: al- les zu wer- den, strömt zu- hauf! Völ- ker,
    "G"G4 D3 B, | "C"E4 C2 "(Am)"A>G | "D"F4 E3 D | "G"D4 z2 D | B3 B "D"A2 D2 |
w: hört die Sig- na- le! Auf zum letz- ten Ge- fecht!
+: Die In- ter- na- tio-
    "G"G4 "D"F3 F | "A"E3 ^D E2 "A7"A2 | "D"A4 z2 B>A | "G"G4 D3 B, | "C"E4 C2 "(Am)"A>G |
w: na- le er- kämpft das Men- schen- recht. Völ- ker, hört die Sig- na- le! Auf zum
"D"F4 E3 D | "G"B4 z2 B | d3 d c2 B2 | "D"(A2B2) c2 z c | "G"B3 G "D"A3 F | "G"G4 z2 |]
w: letz- ten Ge- fecht!
+: Die In- ter- na- tio- na-- le
+: er- kämpft das Men- schen- recht.

\end{abc}

\begin{guitarMagic}
    \begin{enumerate}
        \setcounter{enumi}{1} % Starte bei 2 da die erste Strophe bei den Noten steht.
        \liedweiter

        \item Es [G]rettet uns kein höh’res W[C]e[(AM)]sen, 
            kein G[D]ott, kein K[D7]aiser noch Tri[G]bun \\
            Uns [G]aus dem Elend zu erl[C]{\"o}[(Am)]sen 
            können wi[D]ir nur s[D7]elber t[G]un! \\
            Leeres W[D]ort: des [A7]Armen R[D]echte, 
            Leeres W[D]ort: des [A7]Reichen Pfl[D]icht! \\
            Unm[D7]{\"u}ndig nennt man uns und Kn[G]echte, 
            duldet die Schm[D]ach nun l[A7]{\"a}nger n[D]icht! [D7]

        \textbf{Refrain:}
            \leftrepeat Völker, h[G]{\"o}rt die Sign[C]a[(Am)]le!
            Auf zum l[D]etzten Gef[G]echt! \\
            Die Intern[D]ation[G]al[D]e
            erk[G]{\"a}mpft das M[D]enschenr[G]echt. \rightrepeat


        \item In St[G]adt und Land, ihr Arbeitsl[C]e[(Am)]ute,
            w[D]ir sind die st[D7]{\"a}rkste der Part[G]ei’n \\
            Die M[G]{\"u}{\ss}iggänger schiebt beis[C]e[(Am)]ite! 
            Diese W[D]elt muss [D]unser s[G]ein; \\
            Unser Bl[D]ut s[A7]ei nicht mehr der R[D]aben, 
            Nicht der m[A]{\"a}cht’gen G[A7]eier Fr[D]aß! \\
            Erst w[D7]enn wir sie vertrieben h[G]aben 
            dann scheint die S[D]onn’ ohn’ [A7]Unterl[D]ass! [D7]

        \textbf{Refrain:}
            \leftrepeat Völker, h[G]{\"o}rt die Sign[C]a[(Am)]le!
            Auf zum l[D]etzten Gef[G]echt! \\
            Die Intern[D]ation[G]al[D]e
            erk[G]{\"a}mpft das M[D]enschenr[G]echt. \rightrepeat


    \end{enumerate}
\end{guitarMagic}

\notebox{Die Internationale ist ein sozialistisches Arbeiterlied. Ursprünglich wurde das Lied auf französisch geschrieben und später ins Deutsche übersetzt.}

