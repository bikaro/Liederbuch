\def\Titel{Die Internationale}
\def\Interpret{Melodie: Pierre De Geyter (1888)  Text: Emil Luckhardt (1919)}
\def\Referenz{Möglicher Querverweis auf ein gebräuchliches Liederbuch deiner Wahl}

% hier entweder \LiedSetup() oder \GeistlichSetup{} wenn es im Kuratenverzeichnis auftauchen soll
\LiedSetup{}

\begin{abc}[name=DieInternationale,program={abcm2ps -O= --pagescale 1.1}]

X: 1
M: C % Takt
K: G % Tonart
D2 | G3 F A G D B, | E4 C2 E2 | A3 G F E D C | B,4 z2 D |   % Noten
w: Wacht auf, Ver- damm- te die- ser Er- de,                % Text
+: die stets man noch zum Hun- gern zwingt! Das
G3 F A G D B, | E4 C2 a G | F2 A2 c2 F2 | G4 z2 B A |
w: Recht wie Glut im Kra- ter- her- de
+: nun mit Macht zum Durch- bruch dringt. Rei- nen
F3 F (EF) G E | F4 D2 ^C D | E3 E A3 G | F4 z2 A2 |
w: Tisch macht mit_ dem Be- drän- ger!
+: Heer der Skla- ven, wa- che auf! Ein
A3 G D D ^C D | B4 G E F G | F2 A2 G2 E2 | F4 z2 B>A |
w: Nichts zu sein, tragt es nicht län- ger,
+: al- les zu wer- den, strömt zu- hauf! Völ- ker,
G4 D3 B, | E4 C2 A>G | F4 E3 D | D4 z2 D | B3 B A2 D2 |
w: hört die Sig- na- le! Auf zum letz- ten Ge- fecht!
+: Die In- ter- na- tio-
G4 F3 F | E3 ^D E2 A2 | A4 z2 B>A | G4 D3 B, | E4 C2 A>G |
w: na- le er- kämpft das Men- schen- recht. Völ- ker, hört die Sig- na- le! Auf zum
F4 E3 D | B4 z2 B | d3 d c2 B2 | (A2B2) c2 z c | B3 G A3 F | G4 z2 |]
w: letz- ten Ge- fecht!
+: Die In- ter- na- tio- na-- le
+: er- kämpft das Men- schen- recht.

\end{abc}

\begin{guitarMagic}
    \begin{enumerate}
        \setcounter{enumi}{2} % Starte bei 2 da die erste Strophe bei den Noten steht.
        \liedweiter

        \item Es rettet uns kein höh’res Wesen, 
        kein Gott, kein Kaiser noch Tribun \\
        Uns aus dem Elend zu erlösen 
        können wir nur selber tun! \\
        Leeres Wort: des Armen Rechte, 
        Leeres Wort: des Reichen Pflicht! \\
        Unmündig nennt man uns und Knechte, 
        duldet die Schmach nun länger nicht! 

        \textbf{Refrain:}
        |: Völker, hört die Signale!
        Auf zum letzten Gefecht! \\
        Die Internationale
        erkämpft das Menschenrecht. :|


        \item In Stadt und Land, ihr Arbeitsleute,
        wir sind die stärkste der Partei’n \\
        Die Müßiggänger schiebt beiseite! 
        Diese Welt muss unser sein; \\
        Unser Blut sei nicht mehr der Raben, 
        Nicht der mächt’gen Geier Fraß! \\
        Erst wenn wir sie vertrieben haben 
        dann scheint die Sonn’ ohn’ Unterlass!

        \textbf{Refrain:}
        |: Völker, hört die Signale!
        Auf zum letzten Gefecht! \\
        Die Internationale
        erkämpft das Menschenrecht. :| 

    \end{enumerate}
\end{guitarMagic}

